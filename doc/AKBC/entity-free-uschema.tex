\documentclass{article} % For LaTeX2e
\usepackage{naaclhlt2016}
\usepackage{times}
\usepackage{latexsym}
\usepackage{hyperref}
\usepackage{url}
\usepackage{amsmath,amsthm,amsfonts}
\usepackage{multirow}
\usepackage{xspace}
\usepackage{tikz}
\usetikzlibrary{shapes,backgrounds,patterns}
\usepackage{graphicx}
\graphicspath{ {images/} }


\newcommand{\Prob}{\mathbb{P}}
\newcommand{\todo}[1]{{\bf [[}\textcolor{blue}{ todo: #1}{\bf ]]}}
\newcommand{\fix}{\marginpar{FIX}}
\newcommand{\new}{\marginpar{NEW}}

% named as such because `\arg1' apparently isn't valid
\newcommand{\argOne}{\emph{arg1}\xspace}
\newcommand{\argTwo}{\emph{arg2}\xspace}

\newcommand{\citep}[1]{\cite{#1}}
\newcommand{\citet}[1]{\newcite{#1}}



\title{Universal Schema Without Entity Embeddings}

\author{Patrick Verga \& Andrew McCallum \\
    College of Information and Computer Sciences\\
    University of Massachusetts Amherst\\
    % Amherst, MA 01002, USA \\
    \texttt{\{pat, mccallum\}@cs.umass.edu} \\
}

\begin{document}


\maketitle

\begin{abstract}
Universal schema jointly embeds knowledge bases and textual patterns to reason about entities and relations for automatic knowledge base construction and information extraction.
In the past, entity pairs and relations were represented as learned vectors whose compatibility was determined by a scoring function.
However, this leads to cold-start problems where the model is unable to reason about entity pairs and text patterns unseen at train time.
Recently, compositional Universal Schema was proposed to address generalization to unseen text patterns.
In this work we take the next step of removing explicit entity pair representations.
Instead of learning vector representations for each entity pair in our training set, we treat an entity pair as a function of its relation.
In this paper, we experiment with several aggregation functions and demonstrate that we can perform inference using only learned relation representations.
\end{abstract}



% intro + background

\section{Introduction}
\label{introduction}
The goal of automatic knowledge base construction (AKBC) is building a structured knowledge base (KB) of facts using a noisy corpus of raw text evidence, and perhaps an initial seed KB to be augmented~\citep{NELL,yago,freebase}. AKBC supports downstream reasoning at a high level about extracted entities and their relations, and thus has broad-reaching applications to a variety of domains.

One challenge in AKBC is aligning knowledge from a structured KB with a text corpus in order to perform supervised learning through \emph{distant supervision}. \emph{Universal schema}~\citep{limin} along with its extensions~\citep{yao2013universal,vector_pra,neelakantan2015compositional,logicmfnaacl15,toutanova2015representing} avoids this issue of alignment by jointly embedding KB relations, entities and text patterns. This allows information to propagate between KB annotation and corresponding textual evidence without explicit sentence-relation alignment.

Previous approaches to universal schema express each text relation as a distinct item to be embedded. This harms its ability to generalize, making it impossible to process inputs not precisely seen at training time. However, for large-scale applications we are interested in generalizing to new text patterns, new entities, and even new domains. We focus on the extreme example of domain adaptation to a completely new language, which may have limited resources or labeled data such as treebanks, and only rarely a KB with adequate coverage. 

This paper leverages universal schema  to train a deep sentence encoder that captures the compositional semantics of textual relations, allowing for prediction on inputs never seen before. We demonstrate the generality of our method by evaluating in a multilingual transfer learning setting, extracting relations from a corpus in a new language with no coverage in an existing KB, requiring only that the entities in the text corpora for two languages overlap, as depicted in Figure \ref{tab:multilingual-corpora}.

We further improve our models by tying a small set of word embeddings across languages using only simple knowledge about word-level translations, learning to embed semantically similar textual patterns from different languages into the same latent space.

In extensive experiments on the TAC Knowledge Base Population (KBP) slot-filling benchmark we perform relation extraction in Spanish with no labeled data or direct Spanish-KB overlap, demonstrating that our approach is well-suited for broad-coverage AKBC in low-resource languages and domains. Interestingly, we also find that joint training with Spanish improves English accuracy. 

\begin{figure}[h!]
\begin{center}
\vspace{-1.9069cm}

\def\firstcircle{(0,0) circle (1.5cm)}
\def\secondcircle{(0:2cm) circle (1.5cm)}
\def\midline{[line width=1pt, dashed] node[label={[label distance=-3cm]-15:in KB}] {} node[label={[label distance=-3.5cm]15:not in KB}] {} (-3,0) -- (5,0)}
\begin{tikzpicture}

	\begin{scope}
	  \clip \firstcircle (-3,3) rectangle (3,3);
	  \fill[pattern=north west lines, pattern color=black!70] \firstcircle;
	\end{scope}
	\begin{scope}
	  \clip \secondcircle (0,0) rectangle (4,4);
	  \fill[pattern=north east lines, pattern color=black!70] \secondcircle;
	\end{scope}
	\begin{scope}
	  \clip \firstcircle;
	  \clip (0.5,0) rectangle (3,2);
	  \fill[white] \secondcircle;
	\end{scope}
	\draw \firstcircle node[above=1.5cm] {English};
	\draw \secondcircle node[above=1.5cm] {Low-resource};
	\draw \midline;

\end{tikzpicture}
\caption{Splitting the entities in a multilingual AKBC training set into parts. We only require that entities in the two corpora overlap. Remarkably, we can train a model for the low-resource language even if entities in the low-resource language do not occur in the KB. \label{tab:multilingual-corpora}}
\end{center}
\vspace{-.4cm}
\end{figure}

\section{Background \label{sec:background}}

AKBC extracts unary attributes of the form (\textit{subject}, \textit{attribute}), typed binary relations of the form (\textit{subject}, \textit{relation}, \textit{object}), or higher-order relations. We refer to subjects and objects as \textit{entities}. This work focuses solely on extracting binary relations, though many of our techniques generalize naturally to unary prediction. Generally, for example in Freebase~\citep{freebase}, higher-order relations are expressed in terms of collections of binary relations.

We now describe prior work on approaches to AKBC. They all aim to predict (\emph{s, r, o}) triples, but differ in terms of: (1) input data leveraged, (2) types of annotation required, (3) definition of relation label schema, and (4) whether they are capable of predicting relations for entities unseen in the training data. Note that all of these methods require pre-processing to detect entities, which may result in additional KB construction errors.

\subsection{Relation Extraction as Link Prediction \label{sec:prediction}}
A knowledge base is naturally described as a graph, in which entities are nodes and relations are labeled edges~\citep{yago,freebase}. In the case of \emph{knowledge graph completion}, the task is akin to link prediction, assuming an initial set of (\emph{s, r, o}) triples. See~\citet{nickel2015review} for a review. No accompanying text data is necessary, since links can be predicted using properties of the graph, such as transitivity. In order to generalize well, prediction is often posed as low-rank matrix or tensor factorization. A variety of model variants have been suggested, where the probability of a given edge existing depends on a multi-linear form~\citep{rescal,DBLP:journals/corr/Garcia-DuranBUG15,bishan,transe,wang2014knowledge,lin2015learning}, or non-linear interactions between $s$, $r$, and $o$~\citep{socherkb}.
Other approaches model the compositionality of multi-hop paths, typically for question answering \citep{bordes2014question,gu2015traversing,neelakantan2015compositional}.

\subsection{Relation Extraction as Sentence Classification}
\label{seq:dist}

Here, the training data consist of (1) a text corpus, and (2) a KB of seed facts with provenance, ie. supporting evidence, in the corpus. Given individual an individual sentence, and pre-specified entities, a classifier predicts whether the sentence expresses a relation from a target schema. To train such a classifier, KB facts need to be aligned with supporting evidence in the text, but this is often challenging. For example, not all sentences containing Barack and Michelle Obama state that they are married. A variety of one-shot and iterative methods have addressed the alignment problem~\citep{bunescu2007learning,distant_supervision,riedel2010modeling,yao2010collective,hoffmann2011knowledge,surdeanu2012multi,min2013distant,zengdistant}.
An additional degree of freedom in these approaches is whether they classify individual sentences or predicting at the corpus level by aggregating information from all sentences containing a given pair of entities before prediction. The former approach is often preferrable in practice, due to the simplicity of independently classifying individual sentences and the ease of associating each prediction with a provenance.

\subsection{Open-Domain Relation Extraction}
\label{sec:openIE}
In the previous two approaches, prediction is carried out with respect to a fixed schema $R$ of possible relations $r$. This may overlook salient relations that are expressed in the text but do not occur in the schema. In response, \textit{open-domain} information extraction (OpenIE) lets the text speak for itself: $R$ contains all possible patterns of text occuring between entities $s$ and $o$~\citep{openie,etzioni2008open,resolver}. These are obtained by filtering and normalizing the raw text. The approach offers impressive coverage, avoids issues of distant supervision, and provides a useful exploratory tool. On the other hand, OpenIE predictions are difficult to use in downstream tasks that expect information from a fixed schema. 

Table~\ref{tab:patterns} provides examples of OpenIE patterns. The examples in row two and three illustrate relational contexts for which similarity is difficult to be captured by an OpenIE approach because of their syntactically complex constructions. This motivates the technique in Section~\ref{sec:encoder}, which uses a deep architecture applied to the raw tokens, instead of rigid rules for normalizing text to obtain patterns.

\begin{table}[h!]
\small
\begin{center}
%\hspace{-1cm}
\begin{tabular}{|p{4.85cm}| p{2.37cm} | }
\hline
%Relation &
Sentence (context tokens italicized) & OpenIE pattern\\ \hline
%per:siblings &
{\bf Khan} \emph{'s younger sister,} {\bf Annapurna Devi}, who later married Shankar, developed into an equally accomplished master of the surbahar, but custom prevented her from performing in public. &  \argOne 's * sister \argTwo \\ \hline

%per:cities\_of\_residence &
A professor emeritus at Yale, {\bf Mandelbrot} \emph{was born in Poland but as a child moved with his family to} {\bf Paris} where he was educated. &  \argOne * moved with * family to \argTwo \\ \hline

%per:cities\_of\_residence &
{\bf Kissel} \emph{was born in Provo, Utah, but her family also lived in} {\bf Reno}. & \argOne * lived in \argTwo \\  \hline
\end{tabular}
\caption{Examples of sentences expressing relations. Context tokens (italicized) consist of the text occurring between entities (bold) in a sentence. OpenIE patterns are obtained by normalizing the context tokens using hand-coded rules. The top example expresses the per:siblings relation and the bottom two examples both express the per:cities\_of\_residence  relation. \label{tab:patterns}}
\end{center}
\vspace{-.4cm}
\end{table}

\subsection{Universal Schema}
When applying Universal Schema~\citep{limin} (USchema) to relation extraction, we combine the OpenIE and link-prediction perspectives.  By jointly modeling both OpenIE patterns and the elements of a target schema, the method captures broader relational structure than multi-class classification approaches that just model the target schema. Furthermore, the method avoids the distant supervision alignment difficulties of Section~\ref{seq:dist}. 

~\citet{limin} augment a knowledge graph from a seed KB with additional edges corresponding to OpenIE patterns observed in the corpus. Even if the user does not seek to predict these new edges, a joint model over all edges may be able to exploit regularities of the OpenIE edges to improve modeling of the labels from the target schema. 

The data still consist of $(s,r,o)$ triples, which can be predicted using link-prediction techniques such as low-rank factorization.~\citet{limin} explore a variety of approximations to the 3-mode $(s,r,o)$ tensor. One such probabilistic model is:
\begin{equation}
\Prob \left((s,r,o)\right) = \sigma\left( u_{s,o}^\top v_r \right), \label{eq:US-prob}
\end{equation}

where  $\sigma()$ is a sigmoid function, $u_{s,o}$ is an embedding of the entity pair $(s,o)$, and $v_r$ is an embedding of the relation $r$, which may be an OpenIE pattern or a relation from the target schema. All of the exposition and results in this paper use this factorization, though many of the modeling techniques we present later could be applied easily to the other factorizations described in ~\citet{limin}. Note that learning unique embeddings for each OpenIE relations does not guarantee that similar patterns, such as the final two in Table~\ref{tab:patterns}, will be embedded similarly.

As with most of the techniques in Section~\ref{sec:prediction}, the data only consist of positive examples of edges. The absence of an annotated edge does not imply that the edge is false. In fact, we seek to predict some of these missing edges as true. ~\citet{limin} employ the Bayesian Personalized Ranking (BPR) approach of~\citet{rendle2009bpr}, which does not explicitly model unobserved edges as negative, but instead seeks to rank the probability of observed triples above unobserved triples.


Recently, \citet{toutanova2015representing} extended USchema to not learn individual pattern embeddings $v_r$, but instead to embed text patterns using a deep architecture applied to word tokens. This shares statistical strength between OpenIE patterns with similar words. We leverage this approach in Section~\ref{sec:encoder}. Additional work has modeled the regularities of multi-hop paths through knowledge graph augmented with text patterns~\citep{pra,pra_second,vector_pra,neelakantan2015compositional}.





% methods
%\section{Training a Sentence Classifier without Alignment \label{sec:uschema}}
\section{Model \label{sec:model}}

\subsection {Universal Schema without Entity Embeddings}

While Compositional Universal Schema addresses reasoning over arbitrary textual patterns, it is still limited to reasoning over entity pairs seen at training time.
\citet{verga2015multilingual} approach this problem by using Universal Schema as a sentence classifier - directly comparing a textual relation to a kb relation to perform relation extraction.
However, this approach is unsatisfactory for two reasons.
The first is that this creates an inconsistency between training and testing, as the model is trained to predict compatibality between entity pairs and relations and not relations directly.
Secondly, it considers only a single piece of evidence while making its prediction.
The learned entity pair can be seen as a sumamrization of all relations for which that entity pair was seen.
\todo{describe that its uscham + aggregation - refer to figures}



\begin{figure}[h]
\caption{Aggragating relation type vectors to form entity pair vector}
\centering
\includegraphics[scale=.68]{aggregate-entity}
\end{figure}


\subsection {Aggregation Functions}
In this work we examine several fairly simple aggregation functions.
\textbf{Mean Relation} creates a single centroid for the entity pair by averaging all of its relation vectors.
While this intuitive makes sense as an approximation for the explicit entity pair representation, averaging large numbers of embeddings can lead to a noisy signal.
The \textbf{Max Relation} represents the entity pair as its most similar relation to the query vector of interest.
This
 TopK Relations
 Dimension-wise Max Pool
   Convolution + Dimension-wise Max Pool


% results
\section{Experimental Results\label{sec:results}}


\subsection {Held out Distant Supervision Data}
MAP over a held out subset of our tac distant supervision data


\begin{table}[h!]
\setlength{\tabcolsep}{4.1pt}
\begin{center}
\begin{tabular}{|lr|}
\hline
\bf Model & MAP \\
\hline\hline
USchema              & 47.9 \\
Comp-USchema         & 47.5  \\
\hline\hline
USchema-TopK         & \bf48.5  \\
USchema-Mean         & 48.4 \\
USchema-Max          & 28.1  \\
Comp-USchema-Mean    & ?  \\
Comp-USchema-Max     & ? \\
Comp-USchema-TopK    & ?  \\
\hline

\hline
\end{tabular}
\caption{MAP over a held out subset of our tac distant supervision data
\label{distant-supervision-table}}
\end{center}
\vspace{-.3cm}
\end{table}


% conclusion
\section{Conclusion}
In this paper we explore the extension of Universal Schema that forgoes exlicit entity pair representations for an aggregatation function over mentions.
This extension allows us to handle all entity pairs - whether they were seen at train time or not - and also gives us a trivial connection to the provenance which made the prediction.
We present prelimanary findings for several aggregation functions including those that act on the individual relation level as well as those that learn to pool over groups of relations.
These prelimenary experiments were carried out on a fairly small dataset, we plan to test this on a much larger data set in the future.

In the future we plan two further extensions to this work.
The first is to improve the aggregation function with a query specific attention that will be able to proporitonally pool all evidence while simultaneously producing a weighting over provenanes.
The second extension will be to incorporate this work with existing work on Compositional Universal Schema creating a fully generalizable Universal Schema able to predict on all text.


\subsubsection*{Acknowledgments}

\bibliography{sources}
\bibliographystyle{naaclhlt2016}

\newpage
%\appendix
%\section{Appendix}

\subsection{Additional Qualitative Results}

Our model jointly embeds KB relations together with English and Spanish text. We demonstrate that plausible textual patterns are embedded close to the KB relations they express. Table \ref{tab:top-tac-patterns} shows top scoring English and Spanish patterns given sample relations from our TAC KB.

\begin{table}[h]
\begin{center}
\hspace*{-20pt}
\begin{tabular}{|p{8.3cm}|}
\hline
\textbf{per:sibling} \\
\hline
   \argOne, seg\'{u}n petici\'{o}n the primeros ministro, \endgraf \hspace{5pt} su hermano gemelo \argTwo  			\\ %\cline{3-3}
  \argOne, sea the principal favorito para esto oficina \endgraf \hspace{5pt}que tambi\'{e}n ambiciona su hermano \argTwo 	\\%\cline{3-3}
  \argOne, y su hermano gemelo, the primeros ministro \argTwo 	\\
\hline
  \argOne, for whose brother \argTwo  		\\%\cline{3-3}
  \argOne inherited his brother \argTwo 	\\%\cline{3-3}
  \argOne on saxophone and brother \argTwo 	\\
\hline\hline
%
\textbf{org:top\_members\_employees} \\
\hline
   \argTwo, presidente y director generales the \argOne  			\\%\cline{3-3}
   	\argTwo, presidente of the negocios especializada \argOne  	\\%\cline{3-3}
   	\argTwo (CIA), the director of the entidad, \argOne 	\\
\hline
 \argTwo, vice president and policy director of the \argOne  		\\%\cline{3-3}
 \argTwo, president of the German Soccer \argOne 	\\%\cline{3-3}
  \argTwo, president of the quasi-official \argOne 	\\
\hline\hline
%%
\textbf{per:alternate\_names} \\
\hline
   \argOne(como tambi\'{e}n son sabido para \argTwo 			\\%\cline{3-3}
   \argTwo-cuyos verdaderos nombre sea \argOne 	\\%\cline{3-3}
   	\argOne  tambi\'{e}n sabido como \argTwo 	\\
\hline
   \argOne aka \argTwo 		\\%\cline{3-3}
   \argOne, who also creates music under the pseudonym \argTwo 	\\%\cline{3-3}
   \argOne( of Modern Talking fame ) aka \argTwo  	\\
\hline\hline
%%
\textbf{per:cities\_of\_residence} \\
 \hline
  \argOne, poblado d\'{o}nde vive \argTwo 			\\%\cline{3-3}
   \argOne, una ciudadano naturalizado american y nacido in \argTwo 	\\%\cline{3-3}
   \argOne, que vive in \argTwo 	\\
\hline
   \argOne was born Jan. \# , \#\#\#\# in \argTwo 		\\%\cline{3-3}
   	\argOne was born on Monday in \argTwo 	\\%\cline{3-3}
   \argOne was born at Keighley in \argTwo 	\\
\hline
\end{tabular}
\caption{Top scoring patterns for both Spanish (top) and English (bottom) given query TAC relations. \label{tab:top-tac-patterns}}
\end{center}
\end{table}

\subsection{Details Concerning Cosine Similarity Computation}
\label{app:cosine}
We measure the similarity between $r_{\text{text}}$ and $r_{\text{schema}}$ by computing the vectors' cosine similarity. However, such a distance is not well-defined, since the model was trained using inner products between entity vectors and relation vectors, not between two relation vectors. The US likelihood is invariant to invertible transformations of the latent coordinate system, since $\sigma\left( u_{s,o}^\top v_r \right) = \sigma\left( (A^\top u_{s,o})^\top A^{-1} v_r \right)$ for any invertible $A$. When taking inner products between two $v$ terms, however, the implicit $A^{-1}$ terms do not cancel out. We found that this issue can be minimized, and high quality predictive accuracy can be achieved, simply by using sufficient $\ell_2$ regularization to avoid implicitly learning an $A$ that substantially stretches the space.

\subsection{Data Pre-processing, Distant Supervision and Extraction Pipeline \label{sec:ds-el}}

We replace tokens occurring less than 5 times in the corpus with UNK and normalize all digits to \# (e.g. Oct-11-1988 becomes Oct-\#\#-\#\#\#\#).
For each sentence, we then extract all entity pairs and the text between them as surface patterns, ignoring patterns longer than 15 tokens.
This results in 48 million English `relations'. In Section~\ref{sec:norm}, we describe a technique for normalizing the surface patterns.
We filter out entity pairs that occurred less than 10 times in the data and extract the largest connected component in this entity co-occurrence graph.
This is necessary for the baseline US model, as otherwise learning decouples into independent problems per connected component.
Though the components are connected when using sentence encoders, we use only a single component to facilitate a fair comparison between modeling approaches.
We add the distant supervision training facts from the RelationFactory system, i.e. 352,236 entity-pair-relation tuples obtained from Freebase and high precision seed patterns.
The final training data contains a set of 3,980,164 (KB and openIE) facts made up of 549,760 unique entity pairs, 1,285,258 unique relations and 62,841 unique tokens.

We perform the same preprocessing on the Spanish data, resulting in 34 million raw surface patterns between entities.
We then filter patterns that never occur with an entity pair found in the English data.  This yields 860,502 Spanish patterns.
Our multilingual model is trained on a combination of these Spanish patterns, the English surface patterns, and the distant supervision data described above.
We learn word embeddings for 39,912 unique Spanish word types.
After parameter tying for translation pairs (Section \ref{sec:tie-words}),  there are 33,711 additional Spanish words not tied to English.


\subsection{Generation of Cross-Lingual Tied Word Types}
\label{sec:word-tying}
We follow the same procedure for generating translation pairs as \cite{mikolov2013}. First, we select the top 6000 words occurring in the lowercased Europarl dataset for each language and obtain a Google translation. We then filter duplicates and translations resulting in multi-word phrases. We also remove English past participles (ending in -ed) as we found the Google translation interprets these as adjectives (e.g.,  `she read the borrowed book' rather than `she borrowed the book') and much of the relational structure in language we seek to model is captured by verbs. This resulted in 6201 translation pairs that occurred in our text corpus. Though higher quality translation dictionaries would likely improve this technique, our experimental results show that such automatically generated dictionaries perform well.


\subsection{Open IE Pattern Normalization}
\label{sec:norm}
To improve US generalization, our US relations use log-shortened patterns where the middle tokens in patterns longer than five tokens are simplified. For each long pattern we take the first two tokens and last two tokens, and replace all $k$ remaining tokens with the number $\log k$. For example, the pattern {\bf Barack Obama} {\it is married to a person named} {\bf Michelle Obama} would be converted to: {\bf Barack Obama} {\it is married [1] person named} {\bf Michell Obama}. This shortening performs slightly better than whole patterns. LSTM and CNN variants use the entire sequence of tokens.





\end{document}
