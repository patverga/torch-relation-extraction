\section{Experimental Results\label{sec:results}}

\subsection{Data and Evaluation}

Our evaluation is done over a set of 41 relations from the TAC KBP competition.
We query freebase to find all positive examples for these relations.
We then join this set of entity pairs with the TAC newswire corpus.
During training we hold out 10,000 positive relations.
We compute MAP over the entire columns for each relation.
We hold out 30,000 kb facts from training, using 10,000 for development and 20,000 for testing.
For each relation, we rank all entity pairs for which that relation is not a training fact.
We then compute mean average precision over the set of relations.


\begin{table}[h!]
\setlength{\tabcolsep}{4.1pt}
\begin{center}
\begin{tabular}{|l|r|r|r|}
\hline
\bf source & facts & relation types & entity pairs \\
\hline\hline
text & 287741 & 55268 & 93033  \\
\hline
kb & 152224 & 41 & 119054 \\
\hline

\hline
\end{tabular}
\caption{Data set stats
\label{data stats}}
\end{center}
\vspace{-.3cm}
\end{table}


\subsection {Results}
Our results show that relation aggregation is competitive with entity pair embeddings and it is possible to have a Universal Schema model without entity embeddings.
We find that the most effective aggregation function in our experiments is the TopK relation strategy.
As discussed in Section \ref{sec:functions}, taking the top k relation types strikes a balance between the two extremes of Mean and Max relations.
While Mean relations is competetive with TopK, this is most likely a result of using a small data set.


%\begin{table}[h!]
%\setlength{\tabcolsep}{4.1pt}
%\begin{center}
%\begin{tabular}{|lr|}
%\hline
%\bf Model & MAP \\
%\hline\hline
%entity pair embeddings      & 47.9 \\
%TopK relations              & \bf48.5  \\
%Mean relations              & 48.4 \\
%Max relation                & 28.1  \\
%\hline
%
%\hline
%\end{tabular}
%\caption{MAP over a held out subset of our tac distant supervision data
%\label{tab:results}}
%\end{center}
%\vspace{-.3cm}
%\end{table}

\begin{table}[h!]
\setlength{\tabcolsep}{4.1pt}
\begin{center}
\begin{tabular}{|lr|}
\hline
\bf Model & MAP \\
\hline\hline
entity pair embeddings      & 34.1 \\
\hline
TopK relations (k=5)       & 32.4  \\
Mean relations              & 32.2 \\
Max relation                & 22.7  \\
max-pool                    & 30.0 \\
%convolution-max-pool        & 28.3 \\
\hline

\hline
\end{tabular}
\caption{MAP over a held out subset of our tac distant supervision data
\label{distant-supervision-table}}
\end{center}
\vspace{-.3cm}
\end{table}


%\begin{table}[h!]
%\setlength{\tabcolsep}{4.1pt}
%\begin{center}
%\begin{tabular}{|lr|}
%\hline
%\bf Model & MAP \\
%\hline\hline
%Comp-USchema         & 47.5  \\
%\hline\hline
%Comp-USchema-Mean    & ?  \\
%Comp-USchema-Max     & ? \\
%Comp-USchema-TopK    & ?  \\
%\hline
%
%\hline
%\end{tabular}
%\caption{compositional models : MAP over a held out subset of our tac distant supervision data
%\label{distant-supervision-table}}
%\end{center}
%\vspace{-.3cm}
%\end{table}
